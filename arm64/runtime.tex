\specchapter{LFI-Arm64 Runtime}

\informative{The runtime is responsible for securely loading and executing programs. To allow for as much flexibility as possible, this document specifies only the bare minimum set of properties necessary for the runtime to be secure.}

\specsection{Terminology}

\specitem
The \term{sandbox} is a 4GiB region of memory, starting at a non-zero 4GiB-aligned address.

\specitem
A \term{page} is a region of memory with a size that is divisible by the host's
\term{page size}.

\informative{On Arm64 the host's page size may be 4KiB, 16KiB, or 64KiB depending on the host.}

\specitem
A runtime may operate in \term{dense mode} or \term{sparse mode}. Unless otherwise specified, a Rule applies to both modes.

\specsection{Virtual Memory Mappings}

\specitem
\label{specitem:page}
A \term{page} may only be readable, read-writable, read-executable, or unmapped
(no permissions).

\informative{This Rule implies that a page can never be marked both writable and executable.}

\specitem
If a \term{page} is executable, its memory contents must pass the LFI-Arm64 verifier.

\specitem
The first \term{page} of the \term{sandbox} must be mapped read-only.

\specitem
In \term{dense mode}, the next 80KiB of the \term{sandbox}, starting after
the first \term{page}, must be unmapped.

\specitem
In \term{dense mode}, the last 80KiB of the \term{sandbox} must be unmapped.

\specitem
In \term{dense mode}, no \term{page} in the final 128MiB of the \term{sandbox} may be executable.

\specitem
In \term{sparse mode}, the 4GiB memory region following the \term{sandbox} must be unmapped.

\specitem
In \term{sparse mode}, the 4GiB memory region preceding the \term{sandbox} must be unmapped.

\specitem
All other \term{pages} in the \term{sandbox} may be marked with any valid permission, as listed in Rule \ref{specitem:page}.

\specitem
The null \term{page} (address 0) must be unmapped (this is an address outside the \term{sandbox}).

\informative{The dense virtual memory layout allows one sandbox to be allocated for every 4GiB of virtual memory. This allows for up to 65,535 sandboxes to be allocated within a standard 48-bit Arm64 userspace (e.g., on Linux).}

\informative{The sparse virtual memory layout allows one sandbox to be allocated for every 8GiB of virtual memory. This allows for up to 32,767 sandboxes to be allocated within a standard 48-bit Arm64 userspace (e.g., on Linux).}

\specsection{Runtime Calls}

\specitem
The first \term{page} of the \term{sandbox} must contain 256 runtime call pointers, each 8 bytes in length. This list starts at index zero within the \term{page}. Each runtime call pointer may point to a valid entrypoint in the runtime, or be zero.

\specitem
Control switches to a runtime call entrypoint when the sandboxed code invokes a runtime call. The \code{x30} register will contain the return address. The runtime should treat this address as \textit{unsanitized}, and should only return to it if it is within the \term{sandbox}. If it is not within the \term{sandbox}, the implementation may raise an exception, or may resume execution anywhere inside the \term{sandbox}.
