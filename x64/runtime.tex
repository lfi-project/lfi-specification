\specchapter{LFI-x64 Runtime}

\informative{The runtime is responsible for securely loading and executing programs. To allow for as much flexibility as possible, this document specifies only the bare minimum set of properties necessary for the runtime to be secure.}

\specsection{Terminology}

\specitem
The \term{sandbox} is a 4GiB region of memory, starting at a non-zero 4GiB-aligned address.

\specitem
A \term{page} is a region of memory with a size that is divisible by the
host's \term{page size}.

\informative{On x64, the host's page size is 4KiB.}

\specsection{Virtual Memory Mappings}

\specitem
\label{specitem:page}
A \term{page} may only be readable, read-writable, read-executable, or unmapped
(no permissions).

\informative{This Rule implies that a page can never be marked both writable and executable.}

\specitem
If a \term{page} is executable, its memory contents must pass the LFI-x64 verifier.

\specitem
The first \term{page} of the \term{sandbox} must be mapped read-only.

\specitem
The 40GiB memory region following the \term{sandbox} must be unmapped.

\specitem
The 40GiB memory region preceding the \term{sandbox} must be unmapped.

\specitem
The null \term{page} (address 0) must be unmapped (this is an address outside the \term{sandbox}).

\informative{The virtual memory layout allows one sandbox to be allocated for every 44GiB of virtual memory. This allows for up to 2,977 sandboxes to be allocated within a standard 47-bit x86-64 userspace (e.g., on Linux).}

\specsection{Runtime Calls}

\specitem
The first \term{page} of the \term{sandbox} must contain 256 runtime call pointers, each 8 bytes in length. This list starts at index zero within the \term{page}. Each runtime call pointer may point to a valid entrypoint in the runtime, or be zero.

\specitem
Control switches to a runtime call entrypoint when the sandboxed code invokes a runtime call. The \code{x30} register will contain the return address. The runtime should treat this address as \textit{unsanitized}, and should only return to it if it is within the \term{sandbox}. If it is not within the \term{sandbox}, the implementation may raise an exception, or may resume execution anywhere inside the \term{sandbox}.
